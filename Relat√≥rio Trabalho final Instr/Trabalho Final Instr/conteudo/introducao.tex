% ----------------------------------------------------------
% Introdução
%Ex: \chapter{TÍTULO A SER IMPRESSO NO CORPO DO TEXTO}{Título no cabeçalho}{Título no Sumario}
% ----------------------------------------------------------
\chapter{INTRODUÇÃO} 
Diariamente nos deparamos com sinais que contém informações sobre algum fenômeno ou acontecimento. Tais sinais precisam ser manipulados e processados para que se tornem utilizáveis, deste modo, foram aplicados os conhecimentos adquiridos na disciplina de instrumentação e processamento de sinais para desenvolver um sistema capaz de gerar sinais com auxílio do gerador de função, para posteriormente filtrá-los e amplificá-los em hardware, possibilitando visualização e análise em software.


\chapter{DESCRIÇÃO DO PROJETO}
Este trabalho propõe o desenvolvimento de um sistema microcontrolado para aquisição e processamento de sinais.

\section{OBJETIVO GERAL}
Implementar um sistema microcontrolado de aquisição para qualquer tipo de sinal utilizando uma frequência de amostragem de 200 Hz. Apresentando toda a instrumentação necessária para a correta visualização em software, dos sinais gerados pelo gerador de função.

\section{OBJETIVOS ESPECÍFICOS}
Como objetivos específicos, destacamos:
 \begin{itemize}
 	\item Projetar circuito elétrico dos filtros e amplificador.
  	\item Esquematizar circuito elétrico do hardware completo.
 	\item Confeccionar hardware do sistema.
 	\item Desenvolver o firmware para processamento e conversão do sinal.
 	\item Desenvolver o software supervisório para visualização e manipulação do sinal.
 	\item Realizar a documentação do projeto.
 \end{itemize}

\chapter{METODOLOGIA}
Este projeto foi executado durante as aulas da disciplina de Instrumentação e Processamento de sinais, onde o grupo foi dividido em equipes de trabalhos, podemos verificar a divisão na Tabela 1. Para a elaboração, foram realizadas pesquisas, seguido do estudo de caso para desenvolvimento do protótipo, tutoriais e vídeos informativos através de pesquisa eletrônica.  
\begin{table}[htb]
	\IBGEtab{ %Este macro alinha o titulo e fonte a esquerda \IBGEtab{caption label} {tabela ou figura} {fonte}
		\caption{Equipes de trabalho}
		\label{tab:esquerda}
	}{
	\begin{tabular}{p{8.3cm}lp{8cm}}
		\hline
		\textbf{Atividades}                      & \textbf{Equipes de trabalho} \\ \hline
		Esquemático do circuito elétrico                        & Eduarda e Patrícia\\ \hline
		Desenvolvimento do hardware do sistema   & Eduarda e Wagner\\  \hline
		Desenvolvimento do firmware              & Johnatan e Wagner \\  \hline
		Desenvolvimento do software supervisório & Arthur\\  \hline
		Documentação do projeto                  & Eduarda e Patrícia\\  \hline
	\end{tabular}
}{
\fonte{os autores.}
%  \nota{Exemplo de nota}
% \nota[Anotações]{Exemplo nota personalizada}
}
\end{table}
