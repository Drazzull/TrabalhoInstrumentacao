
% resumo em português
\begin{resumo}
\noindent
A Inteligência Artificial pode ser caracterizada como sendo um estudo de novos dispositivos computacionais ou de novos métodos para resolução de problemas, afim de imitar ou reproduzir um comportamento inteligente. O objetivo deste trabalho é implementar um sistema inteligente que contenha dois agentes autônomos, que se locomovam em um circuito pré-determinado e realizem o controle de velocidade e distância entre si. O sistema deve ser composto por um software supervisório que seja capaz de apresentar os parâmetros de configuração e monitoramento dos agentes em tempo real, mostrando velocidade e distância entre eles, possibilitando dessa forma, a avaliação do sistema. Para o desenvolvimento deste trabalho foi necessário criar o protótipo dos agentes seguidores de linha, desenvolver o firmware para seguir linha, para controle de distância e velocidade, também foi necessário criar um software supervisório para monitoramento e configuração dos agentes. 
Este trabalho tem como requisito à obtenção de nota na disciplina de Inteligência Artificial II, onde as aulas foram dedicadas para o desenvolvimento do mesmo. Como metodologia adotada, o projeto foi divido em equipes devido grande número de acadêmicos.


 \vspace{0.2cm}
    
 
 Palavras-chave: Inteligência Artificial. Sistema Inteligente. Firmware. Supervisório. 
\end{resumo}

% resumo em inglês
\begin{resumo}[Abstract]	
 	\begin{otherlanguage*}{english}
 	\noindent 
	\textit{
	The Artificial Intelligence can be characterized as being hum new study computational devices or new methods problem resolution in order to imitate or play hum intelligent behavior.
	The objective this work and implement intelligent hum system containing two autonomous agents, which to move about in pre-hum determined circuit and realize the speed control and distance between them. The system must be composed of a monitoring software to be able to display the configuration parameters and monitoring of agents in real time, displaying speed and distance between they, enabling thus, the evaluation system.
	For work this development was necessary to create the prototype of line followers agents, develop firmware to follow you Line, control distance and speed, also it was necessary to create a monitoring software paragraph monitoring and configuration of agents.
	This has work as requirement to note obtaining the Artificial Intelligence II discipline, where were the lessons dedicated to the development do same. As methodology adopted, the project was divided into teams due grande number of academic.
	}
   \vspace{0.2cm}
 
\textit{
    Key-words: Artificial Intelligence. Intelligent System. Firmware. Supervisory.	
}
 	\end{otherlanguage*}
\end{resumo}